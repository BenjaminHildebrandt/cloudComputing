\section{Prerequisites and Scenario}\label{sec:ch1}
% Verlage wollen Geld, Autoren wollen Geld, wir wollen Geld für die Serverkosten.
% Ohne Geld nur Preview, Live-Streaming von Konferenzen und Keynotes, billiger als Ticket + Reise.
% Springer Article 6 Seiten: 40 Euro

Although it is possible to find scientific content and thorough explanations
of complex materia on established platforms like YouTube, it can be
hard to find among other non-science related stuff. Discussion of topics 
is also limited as the comment sections tend to either be full of unanswered questions
or just typical internet madness. By creating a platform for professionals,
that might hopefully change.\\

Monetization is also an important factor as established publishers of journals
seem content to charge around 40 euros for an article, as in: a six-page-excerpt of a journal.
This is a strong indication that enriched content can also be charged for.\\
On Project \textit{SciTube}, Users can create and publish Videos and recorded 
talks and cut out a preview for all other
users. This feature coupled with a rating-system should make it easier 
for interested viewers to decide if they should pay for the whole thing.
Creators can also decide to make the video free for everyone, which 
introduces ads to at least try and cover server costs.\\
% wieviel Werbeeinnahmen bekommt man denn so?
A richer discussion through comments could be made possible through a 
community-based moderation-system similar to platforms like StackExchange.
Users can gain reputation for quality content and proven experts
(say, Doctors and Professors) could contact the team and receive extra reputation
upon proof of academic success.

% Ähnlich Stack Overflow etc: Community-Moderation durch ausgewiesene Fachkräfte, die Reputation sammeln können
% (zb: Interessierte Profs und Doktoren melden sich mit Nachweis an, erhalten direkt +100 Reputation in ihrem Fachgebiet)\\

