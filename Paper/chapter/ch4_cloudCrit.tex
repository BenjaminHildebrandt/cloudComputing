\section{Evaluation of cloud criteria}\label{sec:ch4}
As defined by the US-based \textit{National Institute of Standards and Technology} (NIST) a cloud service must fulfill certain criteria \cite{nistCloud}. This section gives a brief overview and reviews the application's fulfillment of each.
\begin{itemize}
    \item \textbf{On-Demand Self Service:} Users can create and scale resources through the aforementioned (chapter \ref{sec:ch2}) user interfaces at any time and to any capacity they desire. Only starting up and allocating resources takes some time. The implementation uses the Azure API when execute \textit{AzureMediaServices.azcli}.
    \item \textbf{Broad Network Access:} Provided, as Microsoft holds a global server network and content can be delivered as HTML to be viewed in browsers on supporting devices. The access can be expanded by using the in chapter \ref{sec:ch2} proposed CDN component.
    \item \textbf{Resource Pooling:} Resources can be organized in logical groups, but that's not the point. Resources of the provider are shared between multiple users, called 'tenants'. 
    \item \textbf{Rapid Elasticity:} Illusion of infinite resources and scaling: provided by Azure as far as end users have a valid subscription and the cost plan includes scaling. Vertical scaling (scale-up) requires manual action, especially because the shift in price between tiers can be dramatic. Horizontal scaling (scale-out) also depends in some cases from price tiers. Scale-outs usually depend on parameter like CPU, RAM or storage usage (NodeJS webservice). Other service scale totally automatic, as the payment is per GB (BlobStorage).
    \item \textbf{Measured Services:} provided by azure: Insights and Billing, Alerts. Moreover, the implementation specific measurment will be the time, a user watches videos and storage is used by one user.
\end{itemize}
