\documentclass[english]{lni}

\usepackage{url}
\usepackage{hyperref}
\usepackage[utf8]{inputenc}
\usepackage{nameref}
\usepackage{graphicx}

\begin{document}
%\title[Cloud Computing: SciTube] {A scientific Video-on-Demand-Solution powered by Microsoft Azure}
\title{Cloud Computing: SciTube}
\subtitle{A scientific Video-on-Demand-Solution powered by Microsoft Azure}
\author[Markus Herche \and Benjamin Hildebrandt]
{Markus Herche\footnote{Hochschule Fulda, Dept. of Computer Science, Leipziger Straße 123, 36037 Fulda,
Germany \email{markus.j.herche@cs.hs-fulda.de}} \and
Benjamin Hildebrandt\footnote{Hochschule Fulda, Dept. of Computer Science, Leipziger Straße 123, 36037 Fulda,
Germany
\email{benjamin.hildebrandt@cs.hs-fulda.de}}}

\booktitle{Cloud Computing} % Name of book title
\year{2018-2019}


\maketitle

\begin{abstract}
    The saying goes that a picture says more than a thousand words. This should apply to scientific papers as well,
    because why describe e.g. the deformation of a curve, when that can just be shown in a quick video. The following 
    paper outlines a basic cloud-based video-on-demand-solution with a pay-per-view-model. 
\end{abstract}

Note: This project made possible through the Microsoft Azure Academic Program.

\section{Prerequisites and Scenario}
Verlage wollen Geld, Autoren wollen Geld, wir wollen Geld für die Serverkosten\\

Ohne Geld nur Preview, Live-Streaming von Konferenzen und Keynotes, billiger als Ticket + Reise\\

Users can publish Videos and talks and cut out a preview for all users, which together with video ratings should make it easier to decide if you should pay
They can also decide to make it free for everyone, which introduces ads to at least cover our costs\\

Ähnlich Stack Overflow etc: Community-Moderation durch ausgewiesene Fachkräfte, die Reputation sammeln können
(zb: Interessierte Profs und Doktoren melden sich mit Nachweis an, erhalten direkt +100 Reputation in ihrem Fachgebiet)

\section{Microsoft Azure Fundamentals}
Application consists of multiple \glqq building blocks\grqq that can be managed individually.
They belong to a resource group.
Configuring can be done via GUI in any Webbrowser or via Azure CLI scripts
Those can be exported from the GUI for easy reuse, minimizing the risk of missing a checkbox or a specific option

\section{Our Implementation}
Follows the proposed architecture from Microsoft themselves, then again, what else would you add for VOD.
\begin{itemize}
    \item BlobStorage
    \item AppService
    \item Streaming Endpoint
    \item Azure Media Services: Encoding and DRM
\end{itemize}

\section{Financing}
Tabelle mit der Kostenaufstellung von Azure, skalierbare Elemente
in versch. Tiern, also am Anfang nur 100GB BlobStorage, später 1TB, dann 10TB
Kosten am Anfang (Tier 1): ca. 40 Euro im Monat, also mit einem Hartz4-Satz fast ein Jahr lang bezahlbar
Finanzierung wird erst ein Problem, wenn viel mehr Content und Traffic anfallen

Kosten: 80/20 Creator / SciTube
Every view costs: $VideoSize * (Streaming+CDN)$
Every vid costs: $Size * StoragePrice$
Every month costs: $AppServiceBasePrice * Throughput$

\section{Future Work}
CDN can be deployed and used easily but is not really needed right now. 
Scaling needs to be tested under real conditions (1000 -> 100000 Users), but only bc we need to know what to configure
Can assume that it works from a technical standpoint, as MS has way more experience
    
\end{document}
