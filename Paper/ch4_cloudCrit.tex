As defined by the US-based \textit{National Institute of Standards and Technology} (NIST) a cloud service must fulfill
certain criteria \cite{nistCloud}. This section gives a brief overview and reviews the application's fulfillment of each.
\begin{itemize}
    \item \textbf{On-Demand Self Service:} Users can create and scale resources through the aforementioned (ch2) user interfaces at any time and to any capacity they desire. Only starting up and allocating resources takes some time.
    \item \textbf{Broad Network Access:} Provided, as Microsoft holds a global server network and content can be delivered as HTML to be viewed in browsers on supporting devices.
    \item \textbf{Resource Pooling:} Resources can be organized in logical groups, but that's not the point. Resources of the provider are shared between multiple users, called 'tenants'.
    \item \textbf{Rapid Elasticity:} illusion of infinite resources \& scaling: provided by Azure as far as end users are concerned because everyone demands it. so MS has a trained 
    crew steadily working to administrate farms\\
    autoscaling however is only horizontally (more/less instances of the same resource). Vertical scaling (more powerful versions of instances) requires manual action, especially because the shift in price between tiers can be dramatic.
    \item \textbf{Measured Services:} provided by azure: Insights and Billing, Alerts
\end{itemize}
